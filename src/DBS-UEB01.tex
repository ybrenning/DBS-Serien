%! Author = yanni
%! Date = 18.10.2021

\documentclass[a4paper,12pt]{article}
\usepackage{fancyhdr}
\usepackage{fancyheadings}
\usepackage[ngerman,german]{babel}
\usepackage{german}
\usepackage[utf8]{inputenc}
%\usepackage[latin1]{inputenc}
\usepackage[active]{srcltx}
\usepackage{algorithm}
\usepackage[noend]{algorithmic}
\usepackage{amsmath}
\usepackage{amssymb}
\usepackage{amsthm}
\usepackage{bbm}
\usepackage{enumerate}
\usepackage{graphicx}
\usepackage{ifthen}
\usepackage{listings}
\usepackage{struktex}
\usepackage{hyperref}
\usepackage[T1]{fontenc}

%%%%%%%%%%%%%%%%%%%%%%%%%%%%%%%%%%%%%%%%%%%%%%%%%%%%%%
%%%%%%%%%%%%%% EDIT THIS PART %%%%%%%%%%%%%%%%%%%%%%%%
%%%%%%%%%%%%%%%%%%%%%%%%%%%%%%%%%%%%%%%%%%%%%%%%%%%%%%
\newcommand{\Fach}{Datenbanksysteme I}
\newcommand{\Name}{Yannick Brenning}
\newcommand{\Seminargruppe}{C}
\newcommand{\Matrikelnummer}{3732848}
\newcommand{\Semester}{WiSe 21/22}
\newcommand{\Uebungsblatt}{1} %  <-- UPDATE ME
%%%%%%%%%%%%%%%%%%%%%%%%%%%%%%%%%%%%%%%%%%%%%%%%%%%%%%
%%%%%%%%%%%%%%%%%%%%%%%%%%%%%%%%%%%%%%%%%%%%%%%%%%%%%%

\setlength{\parindent}{0em}
\topmargin -1.0cm
\oddsidemargin 0cm
\evensidemargin 0cm
\setlength{\textheight}{9.2in}
\setlength{\textwidth}{6.0in}

%%%%%%%%%%%%%%%
%% Aufgaben-COMMAND
\newcommand{\Aufgabe}[1]{
        {
        \vspace*{0.5cm}
        \textbf{Aufgabe #1}
        \vspace*{0.2cm}
    }
}
%%%%%%%%%%%%%%
\hypersetup{
    pdftitle = {\Fach{}: Übungsblatt \Uebungsblatt{}},
    pdfauthor = {\Name},
    pdfborder = {0 0 0}
}

\lstset{ %
    language=java,
    basicstyle=\footnotesize\tt,
    showtabs=false,
    tabsize=2,
    captionpos=b,
    breaklines=true,
    extendedchars=true,
    showstringspaces=false,
    flexiblecolumns=true,
}

\title{Übungsblatt \Uebungsblatt{}}
\author{\Name{}}

\begin{document}
    \thispagestyle{fancy}
    \lhead{\Fach{} \\ \small \Name{} - \Matrikelnummer{}}
    \rhead{\Semester{} \\  Übungsgruppe \Seminargruppe{}}
    \vspace*{0.2cm}
    \begin{center}
        \LARGE \textbf{Übungsblatt \Uebungsblatt{}}
    \end{center}
    \vspace*{0.2cm}

%%%%%%%%%%%%%%%%%%%%%%%%%%%%%%%%%%%%%%%%%%%%%%%%%%%%%%
%% Insert your solutions here %%%%%%%%%%%%%%%%%%%%%%%%
%%%%%%%%%%%%%%%%%%%%%%%%%%%%%%%%%%%%%%%%%%%%%%%%%%%%%%

    \Aufgabe{1}
        \begin{itemize}

        \item DB, DBVS/DBMS, DBS
        \newline Ein \emph{DBS}, auch genannt Datenbanksystem ist ein System, welches zur dauerhaften Speicherung
        und Verwaltung großer Datenmengen verwendet wird. Ein DBS besteht aus der \emph{DB} (Datenbank) und einem
        \emph{DBVS} Datenbankverwaltungssystem. Die DB enthält hierbei die Menge der gespeicherten Daten, und das
        DBVS ist ein vereinfachtes Software-System, mit dem man diese Daten verwalten und verarbeiten kann.
        Datenbanksysteme werden in Kombination mit Anwendungen und Benutzerschnittstellen verwendet,
        um Informationssysteme zu bilden.
        % Ein naheliegendes Beispiel für ein IS mit einer DBS ist ein Uni-IS,
        % welches in seinen DBS u.a. Daten zu Studenten, Vorlesungen, Dozenten, usw. speichert.
        % Anwendungen könnten hierbei die Erstellung von Stundenplänen und Hörsaalbelegungen sein, sowie Prüfungspläne.
        % Eine Benutzerschnittstelle wäre bspw. das AlmaWeb, wobei Benutzer sich für Studiengänge bewerben, sowie für
        % Module und Veranstaltungen anmelden können.

        \item Mehrnutzerbetrieb, Transaktion, log. Einbenutzerbetrieb, Synchronisation
        \newline Eine \emph{Transaktion} ist eine Folge von DB-Operationen, die eine Manipulation der DB erreichen.
        Eine Transaktion erfüllt die sog. ACID-Eigenschaften, welche verscheidene Arten and Datenintegrität implizieren.
        Dazu zählt die Ablaufintegrität, welche einen kontrollierten Mehrbenutzerbetrieb gewährleisten soll.
        Bei einem \emph{Mehrbenutzerbetrieb} werden Transaktionen parallel von mehreren Nutzern ausgeführt.
        Hierbei ist die Isolation, die dritte ACID-Eigenschaft zu gewährleisten durch den \emph{log. Einbenutzerbetrieb}.
        Dieser setzt voraus, dass trotz mehrerer Benutzer die einzelnen Transaktionen logisch getrennt verarbeitet werden.
        Diese Transaktionen führen zu Änderungen durch mehrere Benutzer, welche dann durch \emph{Synchronisation}
        umgesetzt werden sollen, ohne dass die Nutzer sich dabei gegenseitig behindern.

        \item OLTP, OLAP, Data Warehouse
        \newline \emph{OLTP} steht für Online Transaction Processing und ist am Betrieb eines DBS beteiligt.
        Hierbei handelt es sich um schnelle, meist einfache vorgeplante Abfragen, die auf bestimmte Datensätze zugreifen.
        Die Bearbeitungszeit pro einzelne Transaktion ist kurz, und sie treten häufig auf. Hierbei soll die
        Datenintegrität durch ACID eingehalten werden.
        \emph{OLAP} (Online Analytical Processing) befasst sich im Gegensatz zu OLTP mit großen Datenbeständen,
        z.B. Archivdaten. Das Transaktionsvolumen ist vergleichsweise gering und einzelne Abfragen sind komplex.
        Diese können zur Vorbereitung von Geschäftsentscheidungen dienen und könenn mit sog. \emph{Data Warehouses }
        interagieren. Ein Data Warehouse ist ein Datenpool mit internen und externen Daten, welche konsolidiert,
        analysiert und zur Entscheidungsfindung aufbereitet werden.

    \end{itemize}

    \newpage

    \Aufgabe{2}
    \begin{itemize}
        \item Die Ansicht der Artikelliste soll nach dem Lagerbestand sortiert werden:
        \newline Da sich der Lagerbestand fortlaufend ändert, muss auch die Artikelliste laufend sortiert werden.
        \item Durch eine Vergrößerung der Firma müssen jetzt mehrere Personen den Wareneingang einpflegen
        (vorher nur eine Person):
        \newline Aufwand bei Mehrbenutzerbetrieb, die Isolation und Ablaufintegrität zu gewährleisten, sonst können
        gegenseitige Behinderungen auftreten. Es kommt zu einer laufenden redundanten Synchronisation.
        \item Für einen neuen Online-Auftritt mit Bestellmöglichkeit kommt ein separates Programm zum Einsatz,
        welches aus Sicherheits- und Performanzgründen auf einem separaten Rechner mit einer Kopie der Artikelliste arbeitet:
        \newline Wenn sich die Artikelliste ändert, werden diese Änderungen nicht auf dem separaten Rechner synchronisiert.
        \item Für den Online-Auftritt muss die Artikelliste um einige Attribute ergänzt und die Artikelnummerierung erweitert werden:
        \newline Wenn die Änderungen durch das separate Programm stattfinden, wird nur die Kopie der Artikelliste
        geändert und nicht die tatsächliche Artikelliste.
        \item Durch eine Stromunterbrechung fällt der Server aus:
        \newline Änderungen an der Kopie der Artikelliste können verloren gehen.
        \item Durch Firmenfusion müssen nun erheblich mehr Artikel gespeichert werden:
        \newline Es können Probleme bei der Synchronisation der neuen, vergrößerten Datenbestände auftreten.
    \end{itemize}

    \Aufgabe{3}
    \newline Unter dem Begriff \emph{Datenunabhängigkeit} versteht man den Grad an Isolation zwischen Daten und Programmen.
    Ein hoher Grad and Datenunabhängigkeit ist wichtig, um den Wartungsaufwand an Anwendungsprogrammen bei Datenveränderungen
    möglichst gering zu halten. Es gibt physische und logische Datenunabhängigkeit. Physische Datenunabhängigkeit
    beschreibt die Unabhängigkeit von Daten in der physischen Speicherung. Die logische Datenunabhängigkeit
    beschreibt die logische Strukturierung der Daten, welches bspw. durch Sichten erfolgt. Das logische Schema kann geändert
    werden, ohne die physische Ebene zu verändern und umgekehrt, sodass bspw. bei einer Reorganisation der Speicherformen
    oder des Speicherorts keine Änderung des logischen Schemas stattfindet.

    \newpage

    \Aufgabe{4}
    \begin{itemize}
        \item Abteilungsleiter sollen zusätzlich zu den Daten der Angestellten ihrer Abteilung auch die Daten der
        Angestellter anderer Abteilungen, die an einem Projekt ihrer Abteilung mitarbeiten, lesen können.
        \newline Es werden externe Zugriffsrechte geändert, d.h. es handelt sich um eine externe Änderung des DBS.
        \item Der Lesezugriff auf Projekttitel ist langsam und wird durch Anlegen eines Zugriffspfades (Index)
        beschleunigt.
        \newline Die Struktur bleibt erhalten, es wird lediglich eine interne Zugriffsoperation geändert.
        Es handelt sich also um eine interne Änderung des DBS.
        \item Durch Reorganisation der Gehaltsabrechnung werden Angestelltendaten statt in alphabetischer Reihenfolge
        sortiert nach Gehaltsgruppen benötigt.
        \newline Es handelt sich hierbei um eine externe Änderung, da die Liste nur in der Sortierung der Ausgabe geändert wird.
        \item Ein Anwendungsprogramm zur Anzeige der Liste aller mit ihren zugehörigen Angestellten soll hinsichtlich
        des Datenschutzes keine Adressdaten der Angestellten darstellen.
        \newline Da die Anwendungsprogramme extern verwendet werden, handelt es sich um eine externe Änderung, da
        nach dieser Änderung keine Adressdaten durch die Anwendungsprogramme sichtbar sein werden. % externes Schema
        \item Eine Abteilung wird aufgelöst und alle ehemaligen Angestellten werden anderen Abteilungen zugewiesen.
        \newline Es wird kein Schema geändert, da hier nur eine Abteilung aufgelöst und Mitarbeiter neu verteilt bleiben.
        Das Schema bleibt allerdings gleich.
        \end{itemize}

    \newpage

    \Aufgabe{5}
    \newline (a)Die ACID-Eigenschaften (Atomicity, Consistency, Isolation, Durability)
    Alles oder nichts Eigenschaft, konsistente Zustände, log. Einbenutzerbetrieb mit Locking,
    Persistenz der Änderung und REDO-Recovery
    \newline (c) Annahmen / Regeln bzgl DB-Inhalt, welche erfüllt werden müssen
    Beschreiben sinnvolle und zulässige Änderungen der DB
    - Zustandsbedingungen vs Übergangsbedingungen
    Attributbedingungen: Geb.Jahr numerisch, vierstellig
    Satzbedingung: Geb.Jahr < Einstellungsjahr
    \newline Beispielsszenario: Zahlungsverkehr (Überweisung)
    Transaktion umfasst zwei Operationen:
    OP1 = Verringere Kontostand
    OP2 = Erhöhe Kontostand

    Integritätsbedingung: Debit-Konto (Kontostand nicht negativ)
    Transaktion 2 wird bspw. durch Fehler unterbrochen: IB ist verletzt, Stromausfall
    UNDO T2 (Atomicity), ggfs. REDO T1 (Durability)

    \begin{itemize}
        \item Falsch - wegen A wird dem AWP entweder der Zustand vor BOT oder nach EOT zugesichert
        \item Falsch - wegen I sind Daten, die von anderen Transaktionen modifiziert werden, nicht zu lesen
        \item Richtig - durch Definition von Integritiätsbedingungen, welche durch C gewahrt werden
        \item Falsch - wegen D dürfen erfolgreich abgeschlossene Transaktionen nicht rückgängig genacht werden
        \end{itemize}
    \newpage

%%%%%%%%%%%%%%%%%%%%%%%%%%%%%%%%%%%%%%%%%%%%%%%%%%%%%%
%%%%%%%%%%%%%%%%%%%%%%%%%%%%%%%%%%%%%%%%%%%%%%%%%%%%%%
\end{document}