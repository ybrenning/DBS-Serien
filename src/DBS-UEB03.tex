%! Author = yanni
%! Date = 03.12.2021

\documentclass[a4paper,12pt]{article}
\usepackage{fancyhdr}
\usepackage{fancyheadings}
\usepackage[ngerman,german]{babel}
\usepackage{german}
\usepackage[utf8]{inputenc}
%\usepackage[latin1]{inputenc}
\usepackage[active]{srcltx}
\usepackage{algorithm}
\usepackage[noend]{algorithmic}
\usepackage{amsmath}
\usepackage{amssymb}
\usepackage{amsthm}
\usepackage{bbm}
\usepackage{enumerate}
\usepackage{graphicx}
\usepackage{ifthen}
\usepackage{listings}
\usepackage{struktex}
\usepackage{hyperref}
\usepackage[T1]{fontenc}
\usepackage{wasysym}
\usepackage{textcomp}
\graphicspath{{./IMG/}}

%%%%%%%%%%%%%%%%%%%%%%%%%%%%%%%%%%%%%%%%%%%%%%%%%%%%%%
%%%%%%%%%%%%%% EDIT THIS PART %%%%%%%%%%%%%%%%%%%%%%%%
%%%%%%%%%%%%%%%%%%%%%%%%%%%%%%%%%%%%%%%%%%%%%%%%%%%%%%
\newcommand{\Fach}{Datenbanksysteme I}
\newcommand{\Name}{Yannick Brenning}
\newcommand{\Seminargruppe}{C}
\newcommand{\Matrikelnummer}{3732848}
\newcommand{\Semester}{WiSe 21/22}
\newcommand{\Uebungsblatt}{3} %  <-- UPDATE ME
%%%%%%%%%%%%%%%%%%%%%%%%%%%%%%%%%%%%%%%%%%%%%%%%%%%%%%
%%%%%%%%%%%%%%%%%%%%%%%%%%%%%%%%%%%%%%%%%%%%%%%%%%%%%%

\setlength{\parindent}{0em}
\topmargin -1.0cm
\oddsidemargin 0cm
\evensidemargin 0cm
\setlength{\textheight}{9.2in}
\setlength{\textwidth}{6.0in}

%%%%%%%%%%%%%%%
%% Aufgaben-COMMAND
\newcommand{\Aufgabe}[1]{
        {
        \vspace*{0.5cm}
        \textbf{Aufgabe #1}
        \vspace*{0.2cm}
    }
}
%%%%%%%%%%%%%%
\hypersetup{
    pdftitle = {\Fach{}: Übungsblatt \Uebungsblatt{}},
    pdfauthor = {\Name},
    pdfborder = {0 0 0}
}

\lstset{ %
    language=java,
    basicstyle=\footnotesize\tt,
    showtabs=false,
    tabsize=2,
    captionpos=b,
    breaklines=true,
    extendedchars=true,
    showstringspaces=false,
    flexiblecolumns=true,
    xleftmargin=\dimexpr\fboxsep-\fboxrule,
    xrightmargin=\dimexpr\fboxsep-\fboxrule,
    gobble=10
}

\title{Übungsblatt \Uebungsblatt{}}
\author{\Name{}}

\begin{document}
    \thispagestyle{fancy}
    \lhead{\Fach{} \\ \small \Name{} - \Matrikelnummer{}}
    \rhead{\Semester{} \\  Übungsgruppe \Seminargruppe{}}
    \vspace*{0.2cm}
    \begin{center}
        \LARGE \textbf{Übungsblatt \Uebungsblatt{}}
    \end{center}
    \vspace*{0.2cm}

%%%%%%%%%%%%%%%%%%%%%%%%%%%%%%%%%%%%%%%%%%%%%%%%%%%%%%
%% Insert your solutions here %%%%%%%%%%%%%%%%%%%%%%%%
%%%%%%%%%%%%%%%%%%%%%%%%%%%%%%%%%%%%%%%%%%%%%%%%%%%%%%

    \Aufgabe{1: Grundlagen Relationenmodell}
    \begin{enumerate}[(a)]
        \item Ein \emph{Relationsschema} legt die Anzahl und den Typ der Attribute für eine Tabelle
        in einem relationalen Datenbankschema fest. Ein \emph{relationales Datenbankschema} besteht aus mehrerer
        solcher Tabellen welche miteinander verknüpft sind. Jede Zeile einer solchen Tabelle stellt einen Datensatz dar,
        welcher aus einer Reihe an Attributen (den Zeilen der Tabelle) besteht. Hierbei ist eine \emph{Relation} eine
        Menge, welche mit der Tabelle dargestellt wird.

        \item Der \emph{Grad} n einer Relation ist die Anzahl der Spalten, also die Anzahl seiner Attribute.
        Die \emph{Kardinalität} beschreibt die Anzahl der Sätze, also der Zeilen der Tabelle.

        \item Ein Primärschlüssel wird in einer Spalte der Tabelle gespeichert, um jeden Eintrag mit einem
        einmaligen Wert zu identifizieren. Ein Fremdschlüssel ist ebenfalls ein Attribut, welches in Bezug auf den
        Primärschlüssel einer anderen Relation (d.h. eines Attributs einer anderen Tablle) definiert ist.
    \end{enumerate}

    \Aufgabe{2: ERM \textrightarrow Relationenmodell}
    \begin{enumerate}[(a)]
        \item
        \newpage
        \item
        \begin{lstlisting}[language=SQL]
            CREATE TABLE STATION
                (BEZEICHNUNG VARCHAR (50),
                KAPAZITAET INT,
                KNAME VARCHAR (50),
                PRIMARY KEY (BEZEICHNUNG),
                FOREIGN KEY (KNAME) REFERENCES KRANKENHAUS)

            CREATE TABLE KRANKENHAUS
                (KNAME VARCHAR (50),
                ORT VARCHAR (50),
                PRIMARY KEY (KNAME, ORT)
            FOREIGN KEY (BEZEICHNUNG) REFERENCES STATION)

            CREATE TYPE NAME AS (
                TITEL VARCHAR (50)
                VORNAME VARCHAR (50)
                NACHNAME VARCHAR (50)
            ) NOT FINAL;

            CREATE TABLE PATIENT
                (PATNR INT,
                ALTER INT,
                PNAME NAME,
                PRIMARY KEY (PATNR)
                FOREIGN KEY (KARTENNR) REFERENCES CHIPKARTE)

            CREATE TABLE CHIPKARTE
                (KARTENNR VARCHAR(16),
                GUELTIGBIS VARCHAR(12),
                PRIMARY KEY (KARTENNR),
                FOREIGN KEY (PATNR) REFERENCES PATIENT)

        \end{lstlisting}
    \end{enumerate}

%%%%%%%%%%%%%%%%%%%%%%%%%%%%%%%%%%%%%%%%%%%%%%%%%%%%%%
%%%%%%%%%%%%%%%%%%%%%%%%%%%%%%%%%%%%%%%%%%%%%%%%%%%%%%
\end{document}