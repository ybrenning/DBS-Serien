%! Author = yanni
%! Date = 03.12.2021

\documentclass[a4paper,12pt]{article}
\usepackage{fancyhdr}
\usepackage{fancyheadings}
\usepackage[ngerman,german]{babel}
\usepackage{german}
\usepackage[utf8]{inputenc}
%\usepackage[latin1]{inputenc}
\usepackage[active]{srcltx}
\usepackage{algorithm}
\usepackage[noend]{algorithmic}
\usepackage{amsmath}
\usepackage{amssymb}
\usepackage{amsthm}
\usepackage{bbm}
\usepackage{enumerate}
\usepackage{graphicx}
\usepackage{ifthen}
\usepackage{listings}
\usepackage{struktex}
\usepackage{hyperref}
\usepackage[T1]{fontenc}
\usepackage{wasysym}
\usepackage{textcomp}
\graphicspath{{./IMG/}}

%%%%%%%%%%%%%%%%%%%%%%%%%%%%%%%%%%%%%%%%%%%%%%%%%%%%%%
%%%%%%%%%%%%%% EDIT THIS PART %%%%%%%%%%%%%%%%%%%%%%%%
%%%%%%%%%%%%%%%%%%%%%%%%%%%%%%%%%%%%%%%%%%%%%%%%%%%%%%
\newcommand{\Fach}{Datenbanksysteme I}
\newcommand{\Name}{Yannick Brenning}
\newcommand{\Seminargruppe}{C}
\newcommand{\Matrikelnummer}{3732848}
\newcommand{\Semester}{WiSe 21/22}
\newcommand{\Uebungsblatt}{3} %  <-- UPDATE ME
%%%%%%%%%%%%%%%%%%%%%%%%%%%%%%%%%%%%%%%%%%%%%%%%%%%%%%
%%%%%%%%%%%%%%%%%%%%%%%%%%%%%%%%%%%%%%%%%%%%%%%%%%%%%%

\setlength{\parindent}{0em}
\topmargin -1.0cm
\oddsidemargin 0cm
\evensidemargin 0cm
\setlength{\textheight}{9.2in}
\setlength{\textwidth}{6.0in}

%%%%%%%%%%%%%%%
%% Aufgaben-COMMAND
\newcommand{\Aufgabe}[1]{
        {
        \vspace*{0.5cm}
        \textbf{Aufgabe #1}
        \vspace*{0.2cm}
    }
}
%%%%%%%%%%%%%%
\hypersetup{
    pdftitle = {\Fach{}: Übungsblatt \Uebungsblatt{}},
    pdfauthor = {\Name},
    pdfborder = {0 0 0}
}

\lstset{ %
    language=java,
    basicstyle=\footnotesize\tt,
    showtabs=false,
    tabsize=2,
    captionpos=b,
    breaklines=true,
    extendedchars=true,
    showstringspaces=false,
    flexiblecolumns=true,
    xleftmargin=\dimexpr\fboxsep-\fboxrule,
    xrightmargin=\dimexpr\fboxsep-\fboxrule,
    gobble=10
}

\title{Übungsblatt \Uebungsblatt{}}
\author{\Name{}}

\begin{document}
    \thispagestyle{fancy}
    \lhead{\Fach{} \\ \small \Name{} - \Matrikelnummer{}}
    \rhead{\Semester{} \\  Übungsgruppe \Seminargruppe{}}
    \vspace*{0.2cm}
    \begin{center}
        \LARGE \textbf{Übungsblatt \Uebungsblatt{}}
    \end{center}
    \vspace*{0.2cm}

%%%%%%%%%%%%%%%%%%%%%%%%%%%%%%%%%%%%%%%%%%%%%%%%%%%%%%
%% Insert your solutions here %%%%%%%%%%%%%%%%%%%%%%%%
%%%%%%%%%%%%%%%%%%%%%%%%%%%%%%%%%%%%%%%%%%%%%%%%%%%%%%

    \Aufgabe{1: Grundlagen Relationenmodell}
    \begin{enumerate}[(a)]
        \item Ein \emph{Relationsschema} legt die Anzahl und den Typ der Attribute für eine Tabelle
        in einem relationalen Datenbankschema fest. Ein \emph{relationales Datenbankschema} besteht aus mehrerer
        solcher Tabellen welche miteinander verknüpft sind. Jede Zeile einer solchen Tabelle stellt einen Datensatz dar,
        welcher aus einer Reihe an Attributen (den Zeilen der Tabelle) besteht. Hierbei ist eine \emph{Relation} eine
        Menge, welche mit der Tabelle dargestellt wird.

        \item Der \emph{Grad} n einer Relation ist die Anzahl der Spalten, also die Anzahl seiner Attribute.
        Die \emph{Kardinalität} beschreibt die Anzahl der Sätze, also der Zeilen der Tabelle.

        \item Ein Primärschlüssel wird in einer Spalte der Tabelle gespeichert, um jeden Eintrag mit einem
        einmaligen Wert zu identifizieren. Ein Fremdschlüssel ist ebenfalls ein Attribut, welches in Bezug auf den
        Primärschlüssel einer anderen Relation (d.h. eines Attributs einer anderen Tablle) definiert ist.
    \end{enumerate}

    \Aufgabe{2: ERM \textrightarrow Relationenmodell}
    \begin{enumerate}[(a)]
        \item
        Station((Bezeichnung, KName, KOrt):PK, Kapazit"at \\
        (KName, KOrt): FK REF Krankenhaus(Name, Ort)) \\ \\
        Krankenhaus((Name, Ort):PK) \\ \\
        Patient(PatNr:PK, Alter \\
        KartenNr:FK REF Chipkarte(KartenNr) UNIQUE NOT NULL) \\ \\
        Chipkarte(KartenNr:PK, g"ultigBis) \\ \\
        PatientName(PatNr:PK, Titel, Vorname, Nachname \\
        PatNr: FK REF Patient(PatNr)) \\ \\
        PLiegtAufS((PatNr, KName, KOrt, SBez):PK, von, bis \\
        PatNR:FK REF Patient(PatNr), (KName, KOrt):FK REF Krankenhaus(Ort, Name)
        SBez:FK REF Station(Bezeichnung))

        \newpage
        \item
        \begin{lstlisting}[language=SQL]
            CREATE TABLE STATION(
                BEZEICHNUNG VARCHAR (255) NOT NULL,
                KNAME VARCHAR (255) NOT NULL,
                KORT VARCHAR(255) NOT NULL,
                KAPAZITAET INT NOT NULL,
                PRIMARY KEY (BEZEICHNUNG, KNAME, KORT),
                FOREIGN KEY (KNAME, KORT) REFERENCES KRANKENHAUS(NAME, ORT))

            CREATE TABLE KRANKENHAUS(
                NAME VARCHAR (255) NOT NULL,
                ORT VARCHAR (255) NOT NULL,
                PRIMARY KEY (NAME, ORT))

            CREATE TABLE PATIENT(
                PATNR INT PRIMARY KEY,
                ALTER INT NOT NULL,
                KARTENNR INT UNIQUE NOT NULL,
                NAME NAME,
                FOREIGN KEY (KARTENNR) REFERENCES CHIPKARTE(KARTENNR))

            CREATE TABLE CHIPKARTE(
                KARTENNR INT PRIMARY KEY,
                GUELTIGBIS TIMESTAMP NOT NULL)

            CREATE TABLE PATIENTNAME(
                PATNR INT PRIMARY KEY,
                TITEL VARCHAR(100) NOT NULL,
                VORNAME VARCHAR(50) NOT NULL,
                NACHNAME VARCHAR(50) NOT NULL,
                FOREIGN KEY (PATNR) REFERENCES PATIENT(PATNR))

            CREATE TABLE PLIEGTAUFS(
                PATNR INT NOT NULL,
                SBEZ VARCHAR(255) NOT NULL,
                KNAME VARCHAR(255) NOT NULL,
                KORT VARCHAR(255) NOT NULL,
                VON DATE NOT NULL,
                BIS DATE NOT NULL,
                PRIMARY KEY(PATNR, SBEZ, KNAME, KORT),
                FOREIGN KEY(PATNR) REFERENCES PATIENT(PATNR),
                FOREIGN KEY(SBEZ) REFERENCES STATION(BEZEICHNUNG),
                FOREIGN KEY(KNAME, KORT) REFERENCES KRANKENHAUS(NAME, ORT)
            )

        \end{lstlisting}
    \end{enumerate}

%%%%%%%%%%%%%%%%%%%%%%%%%%%%%%%%%%%%%%%%%%%%%%%%%%%%%%
%%%%%%%%%%%%%%%%%%%%%%%%%%%%%%%%%%%%%%%%%%%%%%%%%%%%%%
\end{document}